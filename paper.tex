%\let\pdfcreationdate=\creationdate % pdfx bug workaround when using tectonic to compile
\documentclass{ceurart}
%\usepackage[utf8]{inputenc}
\usepackage[T1]{fontenc}
\usepackage{graphicx}
\usepackage{csquotes}
\usepackage{siunitx}
\usepackage{hyperref}
\usepackage{cleveref}
\usepackage{booktabs}
%\renewcommand\UrlFont{\color{blue}\rmfamily}
%\newcommand\citep[1]{\cite{#1}}
\urlstyle{rm}
\hypersetup{pdfauthor={Konrad Höffner},pdftitle={RickView: Lightweight Standalone Knowledge Graph Browsing Powered by Rust}}

%%
%% One can fix some overfulls
\sloppy

%%
%% Minted listings support 
%% Need pygment <http://pygments.org/> <http://pypi.python.org/pypi/Pygments>
%\usepackage{listings}
%% auto break lines
%\lstset{breaklines=true}

%%
%% end of the preamble, start of the body of the document source.
\begin{document}
\copyrightyear{2024}
\copyrightclause{Copyright for this paper by its authors. Use permitted under Creative Commons License Attribution 4.0 International (CC BY 4.0).}
\conference{D2R2’24: Third International Workshop on Linked Data-driven Resilience Research 2024}
\title{RickView: Lightweight Standalone Knowledge Graph Browsing}
%\titlerunning{Abbreviated paper title}
\author[1]{Konrad Höffner}[%
orcid=0000-0001-7358-3217,
email=konrad.hoeffner@uni-leipzig.de,
url=https://github.com/KonradHoeffner
]
\cormark[1]
\fnmark[1]
\address[1]{Institute for Medical Informatics, Statistics and Epidemiology (IMISE), Leipzig University, Leipzig, Germany}
%% Footnotes
\cortext[1]{Corresponding author.}

%\authorrunning{K. Höffner}

%
\begin{abstract}
The standard backends for RDF browsers are SPARQL endpoints, which allow expressive queries, but they are overengineered for the simple task of browsing.
RickView is an URI dereferenciation tool and RDF browser that follows an alternative approach of directly querying an in-memory dataset bypassing SPARQL.
It is implemented in Rust, which enables a high level of performance comparable to C and C++ while being memory-safe and thread-safe by design.
RickView is published as a single binary, optionally wrapped in a compact container image, without dependencies on external components.
It is adaptable to a given knowledge base via defaults that can be changed in a configuration file and overridden with environment variables.
It supports the binary RDF format Header Dictionary Triples (HDT), which allows querying compressed RDF in memory.
These features enable the stable long-term, low-maintenance, robust, simple and fast publication of multiple RDF knowledge graphs on low-end hardware.
\end{abstract}
\begin{keywords}Linked Data browser \sep RDF visualization.\end{keywords}

\maketitle

\section{Introduction}\label{summary}

Search, browsing and exploration is a critical area in the linked data (LD) life cycle.
Similar to web browsers, where users enter the URL of a website and receive a display of an HTML document, in RDF browsers,
users enter the URL of an RDF resource and receive a human readable representation.
However in contrast to websites, RDF resources are intended to be processed by machines and do not provide a standardized way of being displayed.
%An RDF graph, also called a knowledge graph, property graph or knowledge base, is a set of RDF triples in the form of \emph{(subject, predicate, object)} and typically describes a common domain under a shared namespace, which can be abbreviated with a prefix, such as \texttt{rdf}.
%For example, the RDF vocabulary has a \enquote{hash namespace} (ending with a \texttt{\#}) of \texttt{http://www.w3.org/1999/02/22-rdf-syntax-ns\#},
%which signifies that all resources, like \texttt{rdf:type} and \texttt{rdf:property}, of the vocabulary are described in the same document.
%This is suitable for small graphs that are not intended to be viewed by non-expert users.

RDF browsers on the other hand \emph{resolve URIs} (URLs) of knowledge graphs with \enquote{slash namespaces} (ending with a \enquote{/} character), so that each resource has its associated page.
Ideally, they offer a machine processable RDF serialization as well, based on either content negotiation, POST parameters or URL variants such as \texttt{http://mynamespace/myresource.ttl}.
RDF browsers describe the direct neighborhood of a resource in an RDF graph, that is they list all triples where the given resource is either the subject (a \emph{direct connection}) or the object (an \emph{inverse connection}).

\paragraph{RickView Design Goals}
%RickView is an RDF browser with the following design goals:
\begin{itemize}
\item
  performance: low memory and CPU utilization, fast page load times and high number of handled requests
\item
  robustness: once it compiles, it runs indefinitely
\item
  standalone: no dependence on services such as a SPARQL endpoint or a web server
\item
  adaptability to any small to medium sized knowledge graph via sensible defaults that can be changed in a configuration file and overridden with environment variables
\item
  containerization: offer a compact container image containing a statically linked binary
\item
  simplicity: RickView is minimalistic and only offers browsing of static data
\end{itemize}

\paragraph{Non goals}
RickView is a read-only visualization for static data, does not offer other visualization types~\citep{linkeddatavisualization}, such as graph based, and is not a search engine.
It is not designed for data that does not fit on one machine (\enquote{big data}) or indeed data that does not fit on RAM.
Extremely large knowledge graphs, such as the complete DBpedia, are better served with the traditional SPARQL endpoint paradigm.

\section{Related Work}\label{relatedwork}
Among the various kinds of LOD visualization tools (see \cref{fig:relatedwork}), RickView is an \emph{RDF and LOD browser} that fulfills three of the four criteria defined in~\cite{adaptinglodview} for a \emph{Pubby-class tool}:
It is (1) a URI dereferencing tool, (2) a LOD visualization tool and (3) an RDF browser.
However by design it is \emph{not} (4) a SPARQL endpoint interface because it does not retrieve information about resources from a SPARQL endpoint but instead from its in-memory representation of the dataset.
%Among the listed 
\citep{adaptinglodview} identifies limitations of LodView which are partially fixed in a unmaintained fork~\footnote{\url{https://github.com/ManlyMan1/LodView_Cyrillic}} with the last change being from 2020.
A pull request to integrate into the original LodView\footnote{\url{https://github.com/LodLive/LodView}} was originally planned but has not yet been created.

\begin{tabular}{lll}
\toprule
Feature												&LodView	&RickView\\
Resolution of Cyrillic URIs							&\\
Decoding Cyrillic URIs in Turtle representations	&\\
Support of Cyrillic literals						&\\
User-friendly URLs for RDF representations			&&\\
Support of hash URIs			&&\\
Expanding nested resources			&&\\
Support of RDF collections			&&\\
Pagination of resource property values			&&\\
Support of LATEX math notation			&&\\
\midrule
\bottomrule
\end{tabular}

%\begin{enumerate}
%\item a URI dereferenciation tool, i.e. when a client requests the URI of an RDF resource via HTTP protocol, the tool handles the request and responds with a resource representation;
%\item a LOD visualization tool, i.e. it represents the resource in human-friendly visual form;
%\item a Linked Data or RDF browser, i.e. it allows a user to navigate between RDF resources of an RDF dataset or the entire LOD cloud, by following links in resource representations;
%\end{enumerate}


\begin{figure}
\centering
%\includegraphics[width=0.7\textwidth]{img/relatedwork.pdf}
\caption{Categorization of LOD visualization tools. Source: \cite{adaptinglodview}.}
\label{fig:relatedwork}
\end{figure}

\section{Use Cases}
SNIK/ANNO/HITO ?
-> easier docker setup than virtuoso, less resources, Java 

LinkedSpending
frequent downtimes
HDT

RuThes
-> LodView instance currently shut down because of memory usage
-> test instance of RickView uses ...
\url{http://ruthes.org/download/ruthes-cloud-corrected-2.ttl.zip}

TODO: check presentation of Leipzig Semantic Web Day for reusable materials

\section{Statement of need}\label{statement-of-need}

While initial enthusiasm in the Semantic Web field has led to a large amount of published knowledge graphs, mainstream adoption has stagnated due to a lack of freely available performant, accessible, robust and adaptable tools~\citep{semanticwebreview}.
Instead, limited duration research grants motivate the proliferation of countless research prototypes, which are not optimized for any of those criteria, are not maintained after the project ends and compete for resources on crowded servers if they do not break down completely.
While there are are several existing RDF browsers, they are not optimized for performance.

\section{Implementation}\label{implementation}

The standard back ends for LD projects are \emph{SPARQL endpoints}, which allow expressive SQL-like \emph{SPARQL queries}, however they are overengineered for the simple task of browsing.
For example, the popular Virtuoso Open-Source Edition maps SPARQL to SQL queries on top of a relational database, which is faster than native triplestores like Apache Jena but requires tuning of parameters like memory buffer sizes for optimal resource allocation and the RDF data model can cause large amounts of joins, which negatively impacts query runtime.

RickView instead follows an alternative approach of directly querying an in-memory dataset bypassing SPARQL.
It is implemented in Rust, which enables a high level of performance comparable to C and C++ while being memory-safe and thread-safe by design.
In order to keep the focus on performance and to get a baseline of functionality for performance evaluation, the web page layout is in large parts copied
over from LodView~\citep{lodview,adaptinglodview}.
%
		RickView supports the binary RDF format \emph{Header Dictionary Triples (HDT)}~\citep{hdt2012} which allows querying compressed RDF in memory using the hdt-rs library~\citep{hdtrs}.
		This allows RickView to serve LinkedSpending\footnote{\url{https://linkedspending.aksw.org}}~\citep{linkedspending} v2015, which takes \SI{28}{\giga\byte} as uncompressed N-Triples and \SI{1.4}{\giga\byte} as HDT, using less than \SI{2}{\giga\byte} RAM.
		%
		Additional deployments include HITO\footnote{\url{https://hitontology.eu/ontology}}~\citep{hito}, SNIK\footnote{\url{https://www.snik.eu/ontology}}~\citep{snik},
		 and ANNO\footnote{\url{https://annosaxfdm.de/ontology}}.

		RickView is published under the MIT License at \url{https://github.com/KonradHoeffner/rickview} and as Docker image at \url{ghcr.io/konradhoeffner/rickview}.

		\begin{acknowledgments}
		Thanks to Simon Bin.
		%The author has no competing interests to declare that are relevant to the content of this article.
		\end{acknowledgments}
		%
		%\bibliographystyle{splncs04}
		\bibliography{paper}

\end{document}
